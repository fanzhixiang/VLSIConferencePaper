% Template for VLSI/CAD-2013 paper; to be used with:
%         vlsiconf.sty  - ICASSP/ICIP LaTeX style file, and
%          IEEEbib.bst - IEEE bibliography style file.
% --------------------------------------------------------------------------
\documentclass{article}
\usepackage{vlsiconf,amsmath,epsfig,xeCJK}

\setCJKmainfont[Mapping=tex-text]{STSongti-TC-Light}
\setCJKsansfont[Mapping=tex-text]{STSongti-TC-Light}
\setCJKmonofont[Mapping=tex-text]{STSongti-TC-Light}


% Example definitions.
% --------------------
\def\x{{\mathbf x}}
\def\L{{\cal L}}

% Title.
% ------
\title{An Integrated Circuit Design for Silicon-Nanowire Read Out Circuit}
%
% Single address.
% ---------------
\name{Author(s) Name(s)\thanks{Thanks to XYZ agency for funding.}}
\address{Author Affiliation(s)}
%
% For example:
% ------------
%\address{School\\
%   Department\\
%   Address}
%
% Two addresses (uncomment and modify for two-address case).
% ----------------------------------------------------------
%\twoauthors
%  {A. Author-one, B. Author-two\sthanks{Thanks to XYZ agency for funding.}}
%   {School A-B\\
%   Department A-B\\
%   Address A-B}
%  {C. Author-three, D. Author-four\sthanks{The fourth author performed the work
%   while at ...}}
%   {School C-D\\
%   Department C-D\\
%   Address C-D}
%
\begin{document}
%\ninept
%
\maketitle
%
\begin{abstract}
Put your abstract here.

\end{abstract}
%
\section{Introduction}
\label{sec:intro}

Among many kinds of one-dimensional nanostructure, silicon nanowire(SiNW) has been highly interested in for the feasible integration with integrated electronic devices.
Many researches of fabrications and electrical properties have been done \cite{J1}.
And since the first time it was introduced to the biosensor field in 2001\cite{J2}, a promising candidate for ultra-sensitive, real-time and label-free  sensor device it became.

While some great advances on element structure design were made\cite{J3}, works of systems-level engineering is insufficient.
Mainly because a proper way of signal acquiring is still indefinite.
In this work, a read-out circuit for ion sensing SiNW based on “constant current” idea is proposed.
Some post-simulation results are showed.



\section{Design Description}
\label{sec:rules}

Conventionally, nanowire is treated as a simple resistor with resistance varies with ion concentration.
The read out circuits are targeted on current measurement \cite{J4} or resistance detecting \cite{C5}.
In this work, nanowire is treated as a complete field-effect transistor(FET).
The read out circuit is design for measuring the current variance, which is interpreted into the transconductance of nanowire.
For the reason that: the ion effects are simplified and hypothetically *summed into the changes on transconductance.

Since nanowire is analogous to the MOSFET, not only ions but intrinsic factors affect transconductance, which should be excluded.
A constant current concept is adopted.

\subsection{Constant Current}

For a simple MOSFET, the transconductance(gm) is
\begin{equation}
    \sqrt{2I_{Dsat} (\kappa \mu C_{ox} \frac{W}{L})}
\end{equation}
in strong inversion region and
\begin{equation}
    \frac{\kappa I_{Ds}}{\phi_t}
\end{equation}
in weak inversion region. $\phi_t$ is thermal voltage.
$I_{Dsat}$ can be simplfied to $I_{Ds}$ for a constant $V_{Ds}$.
$\kappa$ is the gate coupling coefficient that is 1 in strong inversion and approximately 0.4 to 0.7 in weak inversion.
The equations show the transconductance of the MOSFET with fixed size can be roughly decided by giving constant drain-to-source current.

\subsection{Architecture}
The constant current structures such as source follower has been applied to several works of ion-sensitive field-effect transistor(ISFET) \cite{J6,C7}, which is a relative of SiNW.
A similar structure is presented here. The structure can switch between two modes: Gate-Source Voltage Tracing Mode (GVT) (Fig1(a)) and Current Variance Measure Mode (CVM) (Fig1(b)).

Operation in GVT is similar to Source follower.
Except the negative feedback doesn’t happens at source end but gate *end through feedback loop circuits.
This mode devotes to set up nanowire when reference ion solution is given.

CVM happens after suitable gate voltage is found in GVT.
The feedback loop is removed and tested solution is given.
The transconductance of nanowire changes which give rise to current varainace signal.
The signal will be amplified and converted into voltage by a series of transimpedance and voltage amplifier.

\section{Circuit Implementation}




% The paper cannot exceed 2 pages, including images. Please do not
% paginate your paper.
% The first paragraph in each section should not be indented, but
% all following paragraphs within the section should be indented as
% these paragraphs demonstrate.
%
% Page size is A4. The top margin and bottom margin are both 25mm.
% The left margin and right margin are 20mm.  All text must be in a
% two-column format and the font size should be larger than 9pt.
% The space between these two columns is 80mm.
%
% Please use vector graphics instead of bitmap pictures for
% illustrations to achieve high paper quality.
%
% Please embed all the used fonts inside the PDF file. The fonts
% for the text part should be Times.
%
% For the reference part, please follow the rules of IEEE \cite{J1}
% \cite{C2}.

% \subsection{Subheadings}
%
% Subheadings should appear in lower case (initial word capitalized) in boldface.

% \subsubsection{Sub-subheadings}
%
% Sub-subheadings, as in this paragraph, are discouraged.

\section{Conclusion}
\label{sec:conclusion}

Put your conclusion here.


% Below is an example of how to insert images. Delete the ``\vspace'' line,
% uncomment the preceding line ``\centerline...'' and replace ``imageX.ps''
% with a suitable PostScript file name.
% -------------------------------------------------------------------------
\begin{figure}
\begin{minipage}[a]{.48\linewidth}
  \centering
\centerline{\epsfig{figure=./PostSim/close_dc_rload.PNG,width=4.0cm}}
  % \vspace{1.5cm}
  \centerline{(a)GVT Mode }\medskip
\end{minipage}
\hfill
\begin{minipage}[b]{0.48\linewidth}
  \centering
% \centerline{\epsfig{figure=image4.ps,width=4.0cm}}
  \vspace{1.5cm}
  \centerline{(b)CVM Mode }\medskip
\end{minipage}
%
\caption{}
\label{fig:res}
%
\end{figure}




% \begin{figure}
%
% \begin{minipage}[b]{1.0\linewidth}
%   \centering
% \centerline{\epsfig{figure=./PostSim/cap.PNG,width=4.0cm}}
%   % \vspace{2.0cm}
%   \centerline{(a) Result 1}\medskip
% \end{minipage}
% %
% \begin{minipage}[b]{.48\linewidth}
%   \centering
% \centerline{\epsfig{figure=./PostSim/close_dc_rload.PNG,width=4.0cm}}
%   % \vspace{1.5cm}
%   \centerline{(b) Results 3}\medskip
% \end{minipage}
% \hfill
% \begin{minipage}[b]{0.48\linewidth}
%   \centering
% % \centerline{\epsfig{figure=image4.ps,width=4.0cm}}
%   \vspace{1.5cm}
%   \centerline{(c) Result 4}\medskip
% \end{minipage}
% %
% \caption{Example of placing a figure with experimental results.}
% \label{fig:res}
% %
% \end{figure}


% To start a new column (but not a new page) and help balance the last-page
% column length use \vfill\pagebreak.
% -------------------------------------------------------------------------
\vfill
\pagebreak


% References should be produced using the bibtex program from suitable
% BiBTeX files (here: strings, refs, manuals). The IEEEbib.bst bibliography
% style file from IEEE produces unsorted bibliography list.
% -------------------------------------------------------------------------
\bibliographystyle{IEEEbib}
\bibliography{references}

\end{document}
